\renewcommand{\chaptername}{Dinámica de los satelites  elementos keplerianos}
\chapter{Dinámica de los satelites  elementos keplerianos}
\markright{Dinámica de los satelites  elementos keplerianos}
\section{Introducción} 
En este capítulo, se introducen los conceptos básicos relacionados con la física de los satelites, cálculo de las orbitas, y su solución para encontrar las posiciones de los satelites. Estas posiciones vienen dadas desde una base de datos, pero el software (gpredict ystellarium), deben calcular las posiciones en base a su última actualización. La materia que se encarga de este estudio, se denomina mecánica órbital. Esta materia, es absolutamente general, ya sea para movimientos de estrellas, satelites naturales y artificiales. Sus principios son dos leyes básicas de la naturaleza: Las tres leyes de newton y las tres leyes de keepler. A partir de estas dos leyes, y utilizando calculo diferencial, puede obtenerse la solución, para las posiciones de los satelites,estrellas,etc, en función del tiempo, a partir de una condición inicial. 


\section{Conceptos básicos}

En esta sección, se abordan los conceptos básicos para la deducción de las tres leyes de kepler. Se omiten los conceptos de cálculo diferencial y vectores en el plano.Si el lector no esta familiarizado con algunos de estos temas, se sugiere revisar literatura al respecto. Además, si el lector, esta familiarizado con las leyes de newton, sistemas de coordenadas, puede saltearse esta sección. 


Para una compresión completa de la mecánica de los objetos celestes(ya sea satelites artificiales, naturales, y estrellas,etc), se requiere en primer lugar un sistema de coordenadas,y un sistema de referencia de tiempo. En general se elige el centro del sistema de coordenadas, como el centro del planeta tierra, ya que la mayor parte de las observaciones se realizan desde la tierra.

\subsection{cinemática}
	Para seguir el movimiento de una particula en el espacio euclideo,se necesita un marco de referencia, que consta de un reloj, y un sistema de coordenadas. En mecanica no relativista, el reloj es universal para todos los sistemas de coordenadas. Dada una particula P, que esta en un instante t, en la posicion $r(t)$ dada por: 
	$r(t)=x(t)i + y(t)j +z(t) k $ 
	 


% vectores 
% conceptos básicos de leyes de newton 
% fuerza coriolis y deducción de la aceleración en mecanica analítica 
% sistemas de referencia 
% derivada del vector rotacion angular 
% leyes de gravitación universal 
\section{Leyes de kepler}
\subsection{Problema de los dos cuerpos}
\subsection{Problema de n cuerpos} 
\section{Parámetros orbitales o keplerianos}
\section{Determinación de las orbitas} 

\section{Involucrando el tiempo en las ecuaciones} 
\section{Calculos de orbitas y modelos matematicos}
\section{estaciones terrenas}