\usepackage[spanish]{babel}% idioma español
\usepackage[utf8]{inputenc}% para escribir correctamente acentos
\usepackage{float}
\usepackage{array}
\usepackage{caption}

%\captionsetup[subfigure]{justification=centering}
\usepackage{subcaption}
\usepackage[onehalfspacing]{setspace} %cambiar interlineado
\usepackage{amsmath}% para hacer referencia a una ecuación incluyendo la sección donde se encuentra
\numberwithin{equation}{section}
\usepackage{amssymb} %fuente especial para matemáticas
\usepackage[colorlinks=true,breaklinks=true]{hyperref} %para poder navegar a travez de nuestras referencias dentro de nuestro documento. Este hipertexto estará resaltado con color
\usepackage{xcolor} % con las siguientes líneas podemos definir el color de las referencias
\usepackage{tikz}
\definecolor{c1}{rgb}{0,0,1} % azul
\usepackage{lipsum}
\definecolor{c2}{rgb}{0,0.3,0.9} % azul clarito
\definecolor{c3}{rgb}{0.3,0,0.9} % rojo azuloso
\hypersetup{ linkcolor={c1}, citecolor={c2}, urlcolor={c3} } % especificamos el color para cada tipo de referencia (imágenes o ecuaciones, citas bibliográficas y paginas de internet
\usepackage{graphicx} % incluir imágenes.
\usepackage{natbib}% paquete para hacer referencias a la bibliografía
\usepackage[nottoc]{tocbibind} %mostrar la bibliografía en la tabla de contenido
\usepackage{enumerate} %para opciones de enumeración de listas (viñetas y todo eso) \usepackage{todonotes} %para poner notas en el documento, las cuales no se veran en el archivo final.
%\usepackage{fancyhdr} %para tener encabezados bonitos en nuestro documento
\usepackage{titlesec}
% configuracion de paquetes  -------------- 
\addto\captionsspanish{
\renewcommand{\partname}{Fase}
%\renewcommand{\chaptername}{Definición de la problemática y planteo de solucion}
%\renewcommand{\thepart}{}
}
\usepackage{minted}
\usepackage{geometry}



\setlength{\textwidth}{150mm}
\setlength{\textheight}{240mm}
%\setlength{\oddsidemargin}{6mm}
%\setlength{\evensidemargin}{28mm}
%\setlength{\topmargin}{-5mm}


\usepackage{sidecap}


\titleformat{\chapter}[display] { \normalfont} { \partname \ \thepart - capítulo \thechapter:  \chaptername}{-6ex} % sep
{
  \rule{\textwidth}{1pt}
  \vspace{-2ex}
  \bfseries
  \centering
  \Huge
} % before-code
[
  \vspace{-3ex}%
 % \rule{\textwidth}{0.2pt}
] 

\titlespacing{\chapter}{0pt}{-40pt}{2cm}


\titleformat{\section}[block]
{\normalfont\bfseries}
{}{0.0pt}{\large{\thesection}  }

\titlespacing{\section}{0pc}{1.2ex plus .9ex minus .2ex}{.2ex}




\usetikzlibrary{matrix,arrows,positioning,shadows,shadings,backgrounds,calc,shapes, tikzmark}
\usepackage{tcolorbox,empheq} 
\tcbuselibrary{skins,breakable,listings,theorems}


\setcounter{secnumdepth}{3}
\setcounter{tocdepth}{2}

%\newcounter{ns}
%\addtocounter{ns}{1} 

%\setcounter{secnumdepth}{2} %para que ponga 1.1.1.1 en subsubsecciones
%\setcounter{tocdepth}{3} % para que ponga subsubsecciones en el indice
\usepackage{wrapfig}
\usepackage{multirow} 
\usepackage{multicol} 
\usepackage{appendix}

\usepackage[flushleft]{threeparttable}
\usepackage{listings}

\usemintedstyle{arduino}





%\AtBeginEnvironment{subappendices}{
%	\renewcommand{\chaptername}{Desarrollo de software realizado al terminar la fase 2}
%	\chapter*{Desarrollo de software realizado al terminar la fase 2}
	
	
%	\counterwithin{figure}{section}
%	\counterwithin{table}{section}
%}



\titleformat{\chapter}[display] { \normalfont} { \partname \ \thepart - capítulo \thechapter:  \chaptername}{-6ex} % sep
{
	\rule{\textwidth}{1pt}
	\vspace{-2ex}
	\bfseries
	\centering
	\Huge
} % before-code
[
\vspace{-3ex}%
% \rule{\textwidth}{0.2pt}
] 
%\usepackage[dvips]{graphicx}

\title{sistema de posicionamiento}
\author{Gaston Valdez}


\renewcommand{\listingscaption}{codigo}